\documentclass[12pt,a4paper]{article}
\title{EBProto Design Proposal}
\author{Chris L. Gebhart\\Phil Colella \\ Brian Van Straalen\\
  Daniel T. Graves\\Terry J. Ligocki}
\usepackage{hyperref}
\usepackage{manual}


\begin{document}
\lstset{language=C++,style=protostyle}
\maketitle

\section{Introduction}
This is a proposal for how to input EB concepts into Proto without
bloating the interface unnecessarily.   The basic framework division
is 
\begin{itemize}
\item The graph and the data on the graph (regular and irregular)
  can be live in Proto.
\item I would eliminate the distinction between multivalued and
  irregular cells.   That was mainly about fortran.   All irregular
  data lives in the sparse container.
\item Geometry generation lives in Chombo.
\item Chombo generates the stencil definitions in the form
  \being{itemize}
  \item The stencil to be used on regular cells in standard Proto form.
  \item The stencil to be used on irregular cells that gets
    transmitted as something that resembles VoFStencils.
  \emd{itemize}
\item This hybrid stencil gets evaluated in Proto on the device.
\end{itemize}


\section{Proto::Graph}
\begin{minipage}{\linewidth}
  \begin{lstlisting}[caption=\code{forall} ,label=lst-forall]

class Volume
{
  //same as volindex except with Point
}
class Area
{
  //same as faceindex except with Point
}

class Graph
{
 public:
  Graph(vector<IrregNode> a_irregularVolumes,
        Box               a_region, 
        Box               a_domain,
        BoxData<int>      a_regIrregCovered);

  vector<Volume> getVolumes(Point a_pt);
  ///lets make direction be signed--this means we can lose Side::LoHiSide
  vector<Area>   getAreas(Volume a_v, int direction);

  std::unordered_set<Point> getIrregularVolumes(Box a_bx);

  private:
  std::unordered_set<Point> m_irregSet;
  BoxData<GraphNode>        regIrregCovered
};
  \end{lstlisting}
\end{minipage}

\section{Proto::VolumeData}

\end{document}
