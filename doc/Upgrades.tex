\documentclass[12pt,a4paper]{article}
\title{Proto Design Upgrades}
\author{Chris L. Gebhart\\Phil Colella \\ Brian Van Straalen\\
  Daniel T. Graves\\Terry J. Ligocki}
\usepackage{hyperref}
\usepackage{manual}


\begin{document}
\lstset{language=C++,style=protostyle}
\maketitle

\section{Prologue}
First there is some usage confusion.  C++11 {\em lambda expressions} or only currently needed when you actually have to capture something.

\section{Fully variadic \code{forall} syntax}

Capture is actually a very mechanical process and repetitive.  Once the code correctly compiles things work like magic, but getting to compilable code requires internalizing a lot of difficult semantics and grammar and the user error messages from compilers is very hard to understand until you have been doing advanced variadic programming for a while.

I also think that we can get arbitrary local variable capture inside the variadic argument list so that \code{forall} can be truly generic.  The declarations in listing \ref{lst-forall} gives a flavor for how we can support any combination of arguments and mixtures of \code{BoxData} objects and non-\code{BoxData} objects.  The return type is inferred from the first item in the list.  I don't quite know what the user wants to happen if they put a non-\code{BoxData} as the first item in the list, and it is quite possible they would want us to consider that a compilation error. For users to specify they want in-place computation then we would likely need a new function name (you can't overload functions based on return types).

\begin{minipage}{\linewidth}
  \begin{lstlisting}[caption=\code{forall} ,label=lst-forall]
 //arbitrary arguments are passed
  template<typename S>
  auto pref(S& s, const Point& p) {return s;}
 // arbitrary arguments do nothing for increment
  template<typename S>
  auto pincrement(S& s) {} 
  template<typename T, int C, int D, int E>
  auto pref(BoxData<T,C,D,E>& bd, const Point& p)
        { return bd.vars(p);}
  template<typename T, int C, int D, int E>
  auto pincrement(BoxData<T,C,D,E>::reference T)
        {return ++T;}
    
 // helper struct
  template<typename First, typename...Srcs>
  struct Forall_ret_type
  {  typedef return_t std::decltype(First); }
 //Current forall signature except inference
 //on the return-type
 template<typename Func, typename... Srcs> 
 forall_ret_type<Srcs...>::return_t
 forall(const Func& a_F,
        const Box& a_box,
        Srcs&... a_srcs);

 template<typename Func, typename... Srcs> 
 forall_ret_type<Srcs...>::return_t
 forall_p(const Func&  a_F,
          const Box&   a_box,
          Srcs&...      a_srcs);

  \end{lstlisting}
\end{minipage}

  Listing \ref{lst-forall} Also demonstrates the return generalization.  Usage of move assignment to handle the declaration and construction of the return object as well as letting the user code elide these definitions.  Type-ness flows from object to object. This works when things sit just so, but will break when you do things like call scalar functions on tensored \code{BoxData}

  \section{ Box inference}

  In most code sections it is possible to infer the correct \code{Box} to operate on from the context of the call site and the instantiated arguments. The \code{Box} argument should be optional to allow for greater user control where needed, but left out when the default rules can be applied.

  Default Rules:
  \begin{itemize}
  \item \code{forall} iterates over all points in argument 1 and returns a \code{BoxData} that would be a clone of argument 1.
  \item \code{Stencil} would return a \code{BoxData} that is the same type as the Stencil target and defined over the maximal \code{Box} that can be supported by the input arguments.
  \end{itemize}

\section{Box \code{grow} and  \code{shrink}}
\begin{lstlisting}[caption=\code{Box} grow and shrink,label=grow]
 // returns a Box such covers the domain of
 // a_stencil given the range a_box
  template<typename T>
Box grow(const Box& a_box,const Stencil<T>& stencil)

// return a Box the defines the range of
// a_stencil given a domain of a_box 
  template<typename T>
Box shrink(const Box& a_box,const Stencil<T>& a_stencil)
\end{lstlisting}

\section{Member functions}
Users writing out the repetitive lambda capture for member functions
is annoying. The clean solution looks like listing \ref{c++14-auto}. This code won't compile in C++11 and requires C++14.
\begin{lstlisting}[label=c++14-auto,caption=C++14 macro for capturing
    arbitrary member functions]
  #define MFUN(f) \
  [this] (auto &&...args) -> decltype (auto) \
{return f (std::forward<decltype (args)> (args)...);}
\end{lstlisting}

I'm pretty certain, however, that a suitable macro can be written which uses \code{std::function} and \code{std::bind} and tail recursion similar to \url{https://gist.github.com/engelmarkus/fc1678adbed1b630584c90219f77eb48}
which outlines a variadic bind.  That looks more complicated than I think we really need, since we don't need a fully general solution.  I don't have time to game this one out.

\section{Example code using Proto augmentations}

\begin{minipage}{\linewidth}
\begin{lstlisting}[caption=Example EulerOp::operator() using changes]
double EulerOp::operator()(
            BoxData<double,NUMCOMPS>& a_Rhs,
            const BoxData<double,NUMCOMPS>& a_U,
            const Box& a_dbx0)
{
    CH_TIME("EulerOp::operator");
    a_Rhs.setVal(0.0);
    double retval;
    auto W_bar =  forall(consToPrim, a_U);
    auto W_ave = m_laplacian(W_bar,1.0/24.0);
    {
      auto U = m_deconvolve(a_U);
      auto W = forall(consToPrim, U);
      forall(retval, waveSpeedBound,W);
      W_ave += W;
    }// you can scope out temporaries.
  
    for (int d = 0; d < DIM; d++)
    {
      //stencil knows centering changes
      auto W_ave_low  = m_interp_L[d](W_ave); 
      auto W_ave_high = m_interp_H[d](W_ave);   
      auto W_ave_f = forall(d, upwindState,W_ave_low,W_ave_high);
      auto F_bar_f = forall(d, getFlux, W_ave_f);
      F_bar_f *= (1./24.);

      W_f = m_deconvolve_f[d](W_ave_f);
      auto F_ave_f = forall(d, getFlux, W_f);
      F_ave_f += m_laplacian_f[d](F_bar_f,1.0/24.0);
      a_Rhs += m_divergence[d](F_ave_f);
    }
    a_Rhs *= -1./s_dx;
    return retval;
}
\end{lstlisting}

\end{minipage}


\end{document}
