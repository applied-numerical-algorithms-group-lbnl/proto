\documentclass[12pt,a4paper]{article}
\title{Adaptive Mesh Refinement Full Approximation Scheme (AMRFAS)}
\author{Chris L. Gebhart}

\usepackage{manual}
\usepackage{amsmath}
\usepackage{algorithm}
\usepackage{algorithmicx}
\usepackage[noend]{algpseudocode}
\algnewcommand\algorithmicforeach{\textbf{for each}}
\algdef{S}[FOR]{ForEach}[1]{\algorithmicforeach\ #1 \algorithmicdo}

\begin{document}
\lstset{language=C++,style=protostyle}
\maketitle

\section{Variables, Notation, and Syntax}

In general, variables which refer to tokens in \libname are written using a \code{monospaced font}. Vector-like objects are written in $\mbf{bold}$, and sets use black-board font (e.g. $\mbb{R}$). Occasionally words are typed in bold-face simply for \textbf{emphasis}. 

\begin{center}
\begin{tabular}{||c c||}
\hline
Variable Name & Variable Definition \\
\hline\hline
$\phi$ & Independent variable, generally a potential\\
$R$ & Right hand side of Poisson's Equation \\
$\rho$ & Input forcing for Poisson's Equation \\
$r$ & Residual \\
$dx, h$ & Spacing in ALL spatial directions \\
$\lambda$ & Relaxation parameter \\
$F^d$ & Flux in direction $d$ \\
$\delta F$ & Flux register contents (e.g. $F_{fine} - F_{coarse}$) \\
$N$ or \code{DIM} & Number of Dimensions \\
$\Omega$ & Subset of Space, usually a Proto::Box \\
$\mathcal{C}(\Omega)$ & AMR coarsening of $\Omega$ \\
$\Gamma$ & Bitbox \\
$L$ & The left hand side operator in Poisson's equation (e.g. $\nabla\cdot F(\phi)$).\\

\hline
\end{tabular}
\end{center}

\section{Data Structures}

In the current implementation of the AMRFAS framework there are 3 levels of data structure: the AMR level (\code{AMRFAS*.H}), the Multigrid level (\code{Multigrid*.H}), and the operator level (\code{BaseOp.H}). The template parameter \code{DATA} corresponds to the patch-level data holder used by the algorithm. At the time of writing, \code{DATA} must be a valid template parameter of CHOMBO's \code{LevelData} class, however it is likely that in the future \code{DATA} will be a PROTO object: either \code{Proto::BoxData} or something derived from it (e.g. for embedded boundary methods). 

In addition to the above structures, the user must define the actual operator object which will be solved by the AMR hierarchy. This operator - which will be referred to as \code{OP} - inherits from \code{BaseOp} and must implement the methods \code{BaseOp::flux} and \code{BaseOp::init} which are purely abstract. 

\subsection{Operator Level}
\code{BaseOp<NCOMP, NGHOST, DATA>} is templated on the number of components in the independent variable, the required number of ghost cells, and the data type used to hold a single patch respectively. \code{BaseOp} also contains the following data members:
\begin{itemize}
\item \code{Proto::Stencil} objects used for prolongation, restriction, interpolation, divergence, and simple flux computation. 
\item An integer defining the refinement ratio between the current and next coarser level. This will take the value of either \code{AMR\_REFRATIO} or \code{MG\_REFRATIO} depending on how the operator is initialized.
\item A \code{Real} representing the grid spacing of this level
\item A \code{Real} representing the relaxation parameter $\lambda$
\item A \code{LevelData<DATA>} used as a temporary for operations utilizing data at two different levels. In particular, this temporary has the grid spacing of the next coarser level but maintains the box layout of the current level. This data member is designated as $temp_C$ in the pseudocode below.
\end{itemize}

\code{BaseOp} implements the following methods:

\begin{algorithm}
\caption{Residual (Single Level)}
\begin{algorithmic}[1]
\Procedure{Residual}{$r, \phi, R$}\Comment{inputs are \code{LevelData<DATA>\&}}
\State $exchange(\phi)$
\ForEach{\textbf{patch} $i$}
	\State $r_i \gets R_i - L(\phi_i)$
\EndFor
\State \textbf{return} $absMax(r)$\Comment{output is usually unused}
\EndProcedure
\end{algorithmic}
\end{algorithm}


\begin{algorithm}
\caption{Relax}
\begin{algorithmic}[1]
\Procedure{Relax}{$r, \phi, R, n$}\Comment{$r, \phi$, and $R$ are \code{LevelData<DATA>\&}}
\For {$i$ \textbf{in} $range(0,n)$}
	\State $exchange(\phi)$
	\State $residual(r, \phi, R)$
	\State $\phi \gets \phi + \lambda * r$
\EndFor
\EndProcedure
\end{algorithmic}
\end{algorithm}

\begin{algorithm}
\caption{Coarsen}
\begin{algorithmic}[1]
\Procedure{Coarsen}{$\phi_C, \phi$}\Comment{inputs are \code{LevelData<DATA>\&}}
\State $temp_C \gets avg(\phi)$
\State $copyTo(temp_C\rightarrow\phi_C)$ \Comment {regrid from temporary}
\EndProcedure
\end{algorithmic}
\end{algorithm}


\begin{algorithm}
\caption{FineCorrection}
\begin{algorithmic}[1]
\Procedure{FineCorrect}{$\phi, \phi_C, \phi_{C0}$}\Comment{inputs are \code{LevelData<DATA>\&}}
\State $\phi_{C0} \gets \phi_C - \phi_{C0}$
\State $copyTo(\phi_{C0}\rightarrow temp_C)$ \Comment {regrid to temporary}
\State $\phi \gets \phi + interp(temp_C)$
\EndProcedure
\end{algorithmic}
\end{algorithm}

\begin{algorithm}
\caption{interpBoundary}
\begin{algorithmic}[1]
\Procedure{bitPoint}{$\Omega, bitRatio$}
\State \textbf{return} $low(\Omega)/bitRatio$
\EndProcedure
\Procedure{getCoarseEdge}{$p, n, \Omega_p^C,  bitRatio$}
\State $\Omega_n \gets Box(n,n)$
\State $\Omega_n^C \gets refine(\Omega_n, bitRatio / refRatio)$
\State $\partial\Omega_{p,n}^C \gets \Omega_n^C \cap \Omega_p^C$
\State \textbf{return} $\partial\Omega_{p,n}^C$
\EndProcedure
\Procedure{interpBoundary}{$\phi, \phi_C$}\Comment{inputs are \code{LevelData<DATA>\&}}
\State $copyTo(\phi_C \rightarrow temp_C)$
\State $\Omega^F \gets domainBox(\phi)$
\State $\Gamma^F\gets \Omega^F / patchSize$
\ForEach {patch $i$}
	\State $p_i \gets bitPoint(box(\phi_i))$ 
	\ForEach { neighor $n_j$ of $p_i$}
		\If {$n_j \ni \Gamma^F$}
			\State $\partial\Omega_{p,n}^C \gets getCoarseEdge(p_i, n_j,box(temp_{C,i}),bitRatio)$
			\State $\phi_i \gets interpBC(temp_{C,i}) \mid \partial\Omega_{p,n}^C$
		\EndIf
	\EndFor
\EndFor
\EndProcedure
\end{algorithmic}
\end{algorithm}

\begin{algorithm}
\caption{Reflux}
\begin{algorithmic}[1]
\Procedure{reflux}{$R_C, \phi_C, \phi, \delta F, k$}\Comment{k is an optional scale}
\State {$\delta F \gets 0$} \Comment {Initialize}
\State{$exchange(\phi)$}
\State{$exchange(\phi_C)$}
\State{$\delta F \gets incrementFine(F(\phi)) \mid \Omega_F$}
\State{$\delta F \gets incrementCoarse(F(\phi_C)) \mid \Omega_C$} \\
\Comment{\code{LevelFluxRegister} computes $\delta F$ as $F_c - \left\langle F_f\right\rangle$, hence the negative sign}
\State{$R_C \gets R_C + \frac{-k}{dx_C}*\nabla\cdot\delta F$}\Comment{e.g. \code{reflux(RC, -k/(refRatio*dx))}}
\EndProcedure
\end{algorithmic}
\end{algorithm}

\begin{algorithm}
\caption{CoarseRhs (Multigrid)}
\begin{algorithmic}[1]
\Procedure{CoarseRhs}{$R_C, \phi_C, \phi, R$}\Comment{inputs are \code{LevelData<DATA>\&}}
\State $exchange(\phi)$
\State $temp_C \gets avg(R - L(\phi))$
\State $copyTo(temp_C\rightarrow R_C)$ \Comment {regrid from temporary}
\State $exchange(\phi_C)$
\State $R_C \gets R_C + L(\phi_C)$
\EndProcedure
\end{algorithmic}
\end{algorithm}

\begin{algorithm}
\caption{Coarse Residual (AMR)}
\begin{algorithmic}[1]
\Procedure{CoarseResidual}{$r_C, \phi_C, R_C, \phi, R, \delta F$}
\State $exchange(\phi_C)$
\State $r_C \gets R_C - L(\phi_C)$
\State $reflux(r_C, \phi_C, \phi, \delta F,-1)$\Comment{negative scale $\rightarrow \delta F = F_c - \left\langle F_f\right\rangle$}
\State $temp_C \gets avg(R - L(\phi))$
\State $copyTo(temp_C\rightarrow R_C)$ \Comment {regrid; cleans up after reflux}
\State \textbf{return} $absMax(r)$\Comment{output is usually unused}
\EndProcedure
\end{algorithmic}
\end{algorithm}

\pagebreak
\subsection{Multigrid Level}

\code{Multigrid<OP, DATA>} implements the multigrid v-cycling with and without boundary condition interpolation. The data structure is templated on the operator to be solved - \code{OP}, a structure that inherits from \code{BaseOp} - and \code{DATA}, the structure type which will hold a patch of data. A \code{Multigrid} object intended for use inside AMR is initialized slightly differently, similar to the behavior of \code{BaseOp}. \code{Multigrid} contains the following member data:

\begin{itemize}
\item \code{m\_level} where 0 is the coarsest 
\item \code{m\_op} an instance of \code{OP} associated with \code{m\_level}.
\item \code{m\_phiC, m\_phiC0, m\_RC} \code{LevelData<DATA>} quantities on the next coarser level computed on this level. Not used on (or allocated for) level 0. 
\item \code{m\_coarser} a recursive \code{Multigrid} instance. Each \code{Multigrid} object controls a single level and computes inputs for the next coarser level (except level 0)
\item \code{m\_phiCAMR} the next coarser AMR level from which we interpolate boundary conditions. (Unused if the \code{Multigrid} is not designated as being part of an AMR framework.)
\end{itemize}

\code{Multigrid} implements the following methods. Note that the \code{*\_RELAX} variables are compile time constants that define the desired number of relaxations when down-cycling, up-cycling, and iterating at the bottom level. These values are defined in \code{AMRUtils.H}

\begin{algorithm}
\caption{VCycle (Non-AMR version)}
\begin{algorithmic}[1]
\Procedure{vcycle}{$\phi, R$}
\If {\code{level == 0}}
	\State {$op.relax(\phi, R, BOTTOM\_RELAX)$}
\Else
	\State {$op.relax(\phi, R, PRE\_RELAX)$}
	\State {$op.coarsen(\phi_C, \phi)$}
	\State {$copyTo(\phi_C\rightarrow \phi_{C0})$}
	\State {$op.coarseRhs(R_C, \phi_C, \phi, R)$}\Comment{$R_C = \langle R - L(\phi) \rangle + L(\phi_C)$}
	\State {$vcycle(\phi_C, R_C)$}\Comment{Call \code{vcycle} on next coarser \code{Multigrid}}
	\State {$op.fineCorrection(\phi, \phi_C, \phi_{C0})$}
	\State {$op.relax(\phi, R, POST\_RELAX)$}
\EndIf
\EndProcedure
\end{algorithmic}
\end{algorithm}

\begin{algorithm}
\caption{VCycle (AMR Version)}
\begin{algorithmic}[1]
\Procedure{vcycle}{$\phi, \phi_C^{AMR}, R$}\\
\Comment{$\phi_C^{AMR}$ is a \code{LevelData} holding the next coarsest level of AMR data.}
\State {$copyTo(\phi_C^{AMR}\rightarrow \phi_{C,temp}^{AMR})$}\Comment{regridding through \code{copyTo}
\State {$op.interpBoundary(\phi, \phi_{C,temp}^{AMR}, level)$} call}
\If {\code{level == 0}}
	\State {$op.relax(\phi, R, BOTTOM\_RELAX)$}
\Else
	
	
	\State {$op.relax(\phi, R, PRE\_RELAX)$}
	\State {$op.coarsen(\phi_C, \phi)$}
	\State {Interpolate boundary of $\phi_C$}
	\State {$copyTo(\phi_C\rightarrow \phi_{C0})$}
	\State {$op.coarseRhs(R_C, \phi_C, \phi, R)$}
	\State {$vcycle(\phi_C, R_C)$}\Comment{Call \code{vcycle} on next coarser \code{Multigrid}}
	\State {$op.fineCorrection(\phi, \phi_C, \phi_{C0})$}
	\State {$op.relax(\phi, R, POST\_RELAX)$}
\EndIf
\EndProcedure
\end{algorithmic}
\end{algorithm}
\pagebreak
\subsection{AMRFAS Level}

The structure of the \code{AMRFAS} object mirrors \code{Multigrid}. \code{AMRFAS} is templated on an operator \code{AMR\_OP} which is effectively the same as the \code{OP} parameter of \code{Multigrid}.

The members of \code{AMRFAS} are as follows:

\begin{itemize}
\item \code{level}, an integer for the AMR level of this object. Level 0 is the coarsest. 
\item \code{mg}, a \code{Multigrid} object used for smoothing
\item \code{op}, an instance of \code{AMR\_OP} defined for use with AMR
\item \code{phi\_C0}, temporary storage for $\phi_C$.
\item \code{RC}, temporary storage for the coarse right-hand side
\item \code{coarser}, the next coarser instance of \code{AMRFAS}.
\item \code{reflux}, an instance of \code{LevelFluxRegister} used for refluxing.  
\end{itemize}

In the following description, a superscript AMR denotes a full AMR hierarchy. The analogous variables without this superscript represent data on the current level (or the next coarser level if there is a subscript C).

\begin{algorithm}
\caption{AMRVCycle}
\begin{algorithmic}[1]
\Procedure{AMRVcycle}{$\phi^{AMR}, \rho^{AMR}, r^{AMR}, R$}\\
\Comment{"AMR" superscript designates an AMR hierarchy variable}
\State $\phi\gets\phi^{AMR}[level]$
\State $\rho\gets\rho^{AMR}[level]$
\State $r \gets r^{AMR}[level]$
\If {\code{level == 0}}
	\State {$op.vcycle(\phi, R)$}\Comment{normal MG V-Cycle}
\Else
	\State $\phi_C\gets\phi^{AMR}[level-1]$
	\State $\rho_C\gets\rho^{AMR}[level-1]$
    \State $r_C\gets r^{AMR}[level-1]$
	 \State {$mg.vcycle(\phi, \phi_C, R)$} \Comment {MG V-Cycle with BC interp}
	 \State {$op.coarsen(\phi_C, \phi)$}
	 \State {$copyTo(\phi_C\rightarrow \phi_{C0})$}
	 \If {\code{level > 1}} \Comment {At least 2 coarser levels exist}
	 	\State {$op.coarsen(\phi_{CC}, \phi_C)$}
	 	\State {$coarser.op.interpBoundary(\phi_C, \phi_{CC})$}
	 \EndIf
	 \State {$op.coarseResidual(r_C, \phi_C, \rho_C, \phi, R, \delta F)$}
	 \State {$R_C \gets r_C + L(\phi_C)$}	 
	 \State {$AMRVcycle(\phi^{AMR}, \rho^{AMR}, r^{AMR}, R_C)$}\Comment {Recursive call}
	 \State {$op.fineCorrection(\phi, \phi_C, \phi_{C0})$}
	 \State {$op.interpBoundary(\phi, \phi_C)$} \Comment {high order interpolation}
	 \State {$mg.vcycle(\phi, \phi_C, R)$} \Comment {MG V-Cycle with BC interp}
	 
	 \State{$op.residual(r, \phi, R)$}\Comment{relevent at finest level}
	 \State{$op.coarseResidual(r_C, \phi_C, \rho_C, \phi, R, \delta F$}
\EndIf

\EndProcedure
\end{algorithmic}
\end{algorithm}

\begin{algorithm}
\caption{AMRApply}
\begin{algorithmic}[1]
\Procedure{AMRApply}{$R^{AMR}, \phi^{AMR}$}\\
\State{$\phi \gets \phi^{AMR}[level]$}
\State{$R \gets R^{AMR}[level]$}
\If {\code{level > 0}}
	\State{$\phi_C \gets \phi^{AMR}[level-1]$}
	\State{$R_C \gets R^{AMR}[level-1]$}
	\State{$AMRApply(R^{AMR}, \phi^{AMR})$} \Comment{recursive call on next coarsest level}
	\State{$R \gets L(\phi)$}
	\State{$reflux(R_C, \phi_C, \phi, \delta F)$}
	\State{$R_C \gets \langle R \rangle \mid \mathcal{C}(\Omega_f)$}
\Else
	\State{$R \gets L(\phi)$}
\EndIf
\EndProcedure
\end{algorithmic}
\end{algorithm}

\end{document}
